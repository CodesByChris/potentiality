% Visualisation of the computation behind the potentiality measure.
%
% Values in visualisation computed as:
%     library(potentiality)
%     library(ghypernet)
%     library(magrittr)
%     library(igraph)
%
%     # Observed network
%     g <- graph(c(1,2,  1,2,  1,2,  1,2,  1,2,  2,3,  2,3), n = 7, directed = FALSE)
%
%     # Alternative networks + likelihoods
%     ens <- ghype(g, directed = FALSE, selfloops = FALSE)
%     dghype <- function(g, ens) {
%         g %>%
%             get.adjacency(sparse = FALSE) %>%
%             logl(xi = ens$xi, omega = ens$omega, directed = ens$directed, selfloops = ens$selfloops) %>%
%             exp() %>%
%             return()
%     }
%     dghype(graph(c(1,2,  1,2,  1,2,  1,2,  1,2,  1,2,  2,3), n = 7, directed = FALSE), ens)  # 0.27
%     dghype(graph(c(1,2,  1,2,  2,3,  2,3,  2,3,  2,3,  2,3), n = 7, directed = FALSE), ens)  # 0.02
%     dghype(graph(c(4,5,  5,6,  5,6,  5,6,  5,6,  5,6,  6,7), n = 7, directed = FALSE), ens)  # 0
%
%     # Potentiality
%     potentiality(g, directed = FALSE, selfloops = FALSE)  # 0.12

\documentclass[border={1cm 0.25cm 1cm 0.75cm}]{standalone}

\usepackage{fontawesome5}
\usepackage{xcolor}
\usepackage{tikz}
\usetikzlibrary{backgrounds}
\usetikzlibrary{calc}
\usetikzlibrary{fit}
\usetikzlibrary{positioning}
\usetikzlibrary{arrows.meta}
\usetikzlibrary{shapes.arrows}
\usetikzlibrary{shapes.geometric}


% newcommands
\newcommand{\code}[1]{\texttt{#1}}


% Colors
\definecolor{potR}{RGB}{168,50,45}
\definecolor{potY}{HTML}{FBB13C}
\definecolor{potB}{HTML}{2E5EAA}



%%%%%%%%%%%%%%%%%%%%%%%%%%%%%%%%%%%%%%%%
%%% Helpers
%%%%%%%%%%%%%%%%%%%%%%%%%%%%%%%%%%%%%%%%

% Draws parallel multi-edges between two specified nodes
%   #1: Node 1
%   #2: Node 2
%   #3: Number of parallel edges
\newcommand{\arrowbundle}[3]{
        % Get starting angle
        \ifnum#3>0  % If #3 is 0 draw no links.
                \def\increment{20}  % i.e. 5 degrees per change
                \pgfmathtruncatemacro\iseven{{Mod(#3,2)}}  % see https://tex.stackexchange.com/a/18790
                \ifnum\iseven=0
                        % Case even number of arrows --> initial turn by \increment/2 degrees
                        \pgfmathsetmacro\baseangle{(\increment / 2) + (((#3 / 2) - 1) * \increment)}
                \else
                        % Case odd number of arrows --> the middle-arrow is at 0 initial turn
                        \pgfmathsetmacro\baseangle{(#3 - 1) / 2 * \increment}
                \fi

                % Draw arrows
                \pgfmathanglebetweenpoints{\pgfpointanchor{#1}{center}}{\pgfpointanchor{#2}{center}}
                \pgfmathsetmacro\abangle{ \pgfmathresult }  % From: https://tex.stackexchange.com/a/39325
                \pgfmathanglebetweenpoints{\pgfpointanchor{#2}{center}}{\pgfpointanchor{#1}{center}}
                \pgfmathsetmacro\baangle{ \pgfmathresult }  % From: https://tex.stackexchange.com/a/39325

                \pgfmathtruncatemacro\lastarrowindex{#3 - 1}
                \foreach \x in {0,...,\lastarrowindex} {
                        \pgfmathsetmacro\currangleab{ \abangle + \baseangle - \x * \increment }
                        \pgfmathsetmacro\currangleba{ \baangle + \baseangle - (\lastarrowindex - \x) * \increment }
                        \draw[Interaction] (#1.\currangleab) -- (#2.\currangleba);
                }
        \fi
}


\newcommand{\nodegrid}{
    \node[Node] (n1) at (0, 0) {};
    \node[Node] (n2) at (2, -0.6) {};
    \node[Node] (n3) at (1, -2) {};
    \node[Node] (n4) at (2.4, -3.5) {};
    \node[Node] (n5) at (0.75, -4) {};
    \node[Node] (n6) at (-0.75, -3.25) {};
    \node[Node] (n7) at (-1.25, -1.75) {};
}



%%%%%%%%%%%%%%%%%%%%%%%%%%%%%%%%%%%%%%%%%%%%%%%%%%%%%%%%%%%%%%%%%%%%%%%%%%%%%%%%
%%% Figure
%%%%%%%%%%%%%%%%%%%%%%%%%%%%%%%%%%%%%%%%%%%%%%%%%%%%%%%%%%%%%%%%%%%%%%%%%%%%%%%%

\begin{document}
    \begin{tikzpicture}
        \pgfdeclarelayer{bgUpper}
        \pgfdeclarelayer{bgLower}
        \pgfsetlayers{bgLower,bgUpper,main}

        \tikzstyle{Node}=[circle, very thick, minimum width=0.75cm, inner sep=0cm, draw=potB, fill=potB!20]
        \tikzstyle{Interaction}=[very thick]
        \tikzstyle{Description}=[align=left, font=\bfseries]
        \tikzstyle{ContinuationDots}=[font=\bfseries\Large, anchor=center]
        \tikzstyle{Box}=[fill=white, draw=black!55, line width=0.15cm, rounded corners=0.7cm, minimum width=8cm, minimum height=9.5cm, inner sep=0.2cm]
        \tikzstyle{Title}=[font=\Large\bfseries]
        \tikzstyle{Subtitle}=[font=\Large]
        \tikzstyle{Arrow}=[-{Triangle[length=5.35cm, width=4.25cm]}, draw=black!12, line width=2.75cm]


        %%%%%%%%%%%%%%%%%%%%%%%%%%%%%%%%%%%%%%%%
        %%% Box 1: Network
        %%%%%%%%%%%%%%%%%%%%%%%%%%%%%%%%%%%%%%%%

        \node[scale=1.4] (Box1) at (0, 0) {
            \begin{tikzpicture}
                \nodegrid

                \arrowbundle{n1}{n2}{5}
                \arrowbundle{n2}{n3}{2}
            \end{tikzpicture}
        };


        %%%%%%%%%%%%%%%%%%%%%%%%%%%%%%%%%%%%%%%%
        %%% Box 2: Ensemble
        %%%%%%%%%%%%%%%%%%%%%%%%%%%%%%%%%%%%%%%%

        \node (Box2) at (10.5, 0) {
            \begin{tikzpicture}

                %%% Nodes
                \node[scale=0.4] (altNet1) at (0, 0) {
                    \begin{tikzpicture}
                        \nodegrid

                        \arrowbundle{n1}{n2}{6};
                        \arrowbundle{n2}{n3}{1};
                    \end{tikzpicture}
                };

                \node[below=0.9cm of altNet1, scale=0.4] (altNet2) {
                    \begin{tikzpicture}
                        \nodegrid

                        \arrowbundle{n1}{n2}{2};
                        \arrowbundle{n2}{n3}{5};
                    \end{tikzpicture}
                };

                \node[below=0.9cm of altNet2, scale=0.4] (altNet3) {
                    \begin{tikzpicture}
                        \nodegrid

                        \arrowbundle{n4}{n5}{1};
                        \arrowbundle{n5}{n6}{5}
                        \arrowbundle{n6}{n7}{1};
                    \end{tikzpicture}
                };

                %%% Dots
                \node[ContinuationDots] at ($(altNet1)!0.5!(altNet2)$) {\vdots};
                \node[ContinuationDots] at ($(altNet2)!0.5!(altNet3)$) {\vdots};
                \node[ContinuationDots, below=0.1cm of altNet3] {\vdots};

                %%% Text
                \node[Description, right=0.9cm of altNet1] {Likely \\ $p\approx 27\%$};
                \node[Description, right=0.9cm of altNet2] {Unlikely \\ $p\approx 2\%$};
                \node[Description, right=0.9cm of altNet3] {Highly unlikely \\ $p\approx 0\%$};
            \end{tikzpicture}
        };


        %%%%%%%%%%%%%%%%%%%%%%%%%%%%%%%%%%%%%%%%
        %%% Box 3: Potentiality
        %%%%%%%%%%%%%%%%%%%%%%%%%%%%%%%%%%%%%%%%

        \node[font=\Large, align=center, text width=6.7cm] (Box3) at (21, 0) {
            \textbf{\code{potentiality() == 0.12}} \\[1cm]

            Close to 0 because very few alternative networks are likely in this example.
        };


        %%%%%%%%%%%%%%%%%%%%%%%%%%%%%%%%%%%%%%%%
        %%% Box Backgrounds
        %%%%%%%%%%%%%%%%%%%%%%%%%%%%%%%%%%%%%%%%

        %%% Boxes
        \begin{pgfonlayer}{bgUpper}
            \node[Box, fill=black!10, draw=black!65, fit=(Box1)] (BoxBackground1) {};
            \node[Box, fill=potY!10, draw=potY!80, fit=(Box2)] (BoxBackground2) {};
            \node[Box, fill=potR!10, draw=potR!80, fit=(Box3)] (BoxBackground3) {};
        \end{pgfonlayer}

        %%% Arrows
        \begin{pgfonlayer}{bgLower}
            \draw[Arrow] (Box1.center) -- (Box2.center);
            \draw[Arrow] (Box2.center) -- (Box3.center);
        \end{pgfonlayer}


        %%%%%%%%%%%%%%%%%%%%%%%%%%%%%%%%%%%%%%%%
        %%% Labels
        %%%%%%%%%%%%%%%%%%%%%%%%%%%%%%%%%%%%%%%%

        \node[Subtitle, below=0.6cm of BoxBackground1.south] (Caption1) {Observed interaction network};
        \node[Subtitle, below=0.6cm of BoxBackground2.south] (Caption2) {Alternative networks};
        \node[Subtitle, below=0.6cm of BoxBackground3.south] (Caption3) {Diversity of alt. networks};

        \node[Title, below=0.12cm of Caption1.south] {\code{igraph::graph()}};
        \node[Title, below=0.12cm of Caption2.south] {\code{ghypernet::ghype()}};
        \node[Title, below=0.12cm of Caption3.south] {\code{potentiality()}};
    \end{tikzpicture}
\end{document}
